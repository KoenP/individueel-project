\documentclass[a4paper,10pt]{article}

\usepackage[dutch]{babel}
\usepackage{graphicx}
\usepackage{amsmath}
\usepackage{amssymb}
\usepackage{amsthm}
\usepackage{wasysym}
\usepackage{float}
\usepackage{listings}
\usepackage{setspace}
\usepackage{tikz}
\usetikzlibrary{arrows}
\onehalfspacing

\DeclareGraphicsExtensions{.pdf}

\begin{document}
\lstset{language=Haskell}
\title{Verslag 2: Lazy evaluation door middel van graafreductie}
\author{Koen Pauwels}
\maketitle

\section{Inleiding}
\subsection{Waarom lazy evaluation?}
\paragraph{}
In de meeste programmeertalen worden argumenten aan een functie ge{\"e}valueerd voordat de functie wordt opgeroepen.
Dit noemen we een \emph{call by value} strategie.
Het is echter mogelijk dat een argument nooit wordt gebruikt in de functie.
In dat geval wordt er overbodig werk geleverd.
\paragraph{}
We kunnen proberen dit overbodig werk te vermijden door de evaluatie van een argument uit te stellen totdat de waarde ervan nodig is.
Dit noemen we een \emph{call by need} of \emph{lazy evaluation} strategie.
Deze strategie is zo goed als uniek voor zuiver functionele talen.
In imperatieve talen berust de betekenis van een programma op het in een correcte, voorspelbare volgorde plaatsvinden van bepaalde \emph{side-effects}.
Met \emph{call by need} kan het erg moeilijk worden om uit te werken in welke volgorde de expressies zullen worden ge{\"e}valueerd, en dus in welke volgorde de side-effects zullen plaatsvinden.
\paragraph{}


\subsection{Waarom een graafvoorstelling?}
Wanneer een $\lambda$-expressie wordt ingeladen in de compiler, zal deze een boomvoorstelling van de expressie genereren.
Zoals eerder vermeld worden argumenten aan functies enkel ge{\"e}valueerd wanneer de waarde nodig is.
Na{\"i}eve normal-order reductie op de boomvoorstelling van de $\lambda$-expressie voldoet wel aan deze voorwaarde, maar zal vaak tot dubbel werk (en dus tot grote ineffici{\"e}nties) leiden.
Wanneer een parameter namelijk meerdere keren voorkomt in het lichaam van een expressie, zal elk voorval van die parameter vervangen moeten worden door een kopie van het argument.
Als we het programma voorstellen als een graaf in plaats van een boom, dan kunnen we elk voorval van de parameter in het lichaam van een $\lambda$-expressie vervangen door een \emph{referentie} naar het argument, in plaats van een kopie ervan.
Bij het evalueren van een argument zullen we ervoor zorgen dat de wortelnode van het argument wordt overschreven met het resultaat van de evaluatie.
Elke referentie naar het argument verwijst dan automatisch en zonder extra kost naar het resultaat van de evaluatie.

TODO: uitleg adhv voorbeelden.


\section{Programmarepresentatie}
De graafvoorstelling moet alle constructies van de $\lambda$-calculus ondersteunen, maar verder verwachten we ook ondersteuning voor een aantal bijzondere builtin operaties en datastructuren.
In theorie is de $\lambda$-calculus voldoende om alle programma's die we wensen uit te drukken, maar in praktijk willen we onder meer de machinevoorstelling gebruiken voor getallen, machine-operaties kunnen uitvoeren op die getallen, gestructureerde datatypes kunnen gebruiken, etc.
\paragraph{}
De basis van de graafvoorstelling is de \emph{cel}, een eenvoudige datastructuur die uit drie elementen bestaat: een \emph{tag} die aangeeft over welk type cel het gaat, en twee \emph{fields}.
Een field kan een pointer naar een andere cel bevatten (op deze manier worden de edges van de graaf ge{\"i}mplementeerd), of een waarde waarvan de betekenis afhangt van het type cel en de context.

\begin{lstlisting}[language=C,frame=single,caption={Definitie van de Cell datastructuur}]
typedef enum {VAR, APP, ABSTR, DATA, BUILTIN, CONSTR} Tag;
union Field {
	struct Cell* ptr;
	char* sym;
	int num;
	Builtin op;
	StructuredDataTag data_tag;
	StructuredDataPtr data_ptr;
};
struct Cell {
	Tag tag;
	union Field f1;
	union Field f2;
};
\end{lstlisting}

De eerste drie celtypes implementeren de structuren van de $\lambda$-calculus, zoals beschreven in \emph{The Implementation of Functional Programming Languages (IFPL)}: variabelen, applicaties en abstracties.
\begin{itemize}
\item
  Een \texttt{VAR} cel gebruikt slechts 1 van zijn fields.
  Dat veld bevat een pointer naar een string die het symbool opslaat.
  Er is een effici{\"e}ntere implementatie mogelijk: we zouden bij het compileren elk symbool kunnen associ{\"e}ren met een uniek getal, en dit getal rechtstreeks in het veld opslaan.
  Ik heb besloten om de originele namen te behouden om debugging niet te bemoeilijken.
\item
  De velden van een \texttt{APP} cel zijn eenvoudigweg celpointers naar de operator en de operand van een applicatie.
\item
  Een \texttt{ABSTR} cel bevat een pointer naar het parameter symbool, en een celpointer naar het lichaam van de functie.
\end{itemize}

Het boek is minder helder over welke extra tags er nodig zijn om de implementatie te vervolledigen.
Mijn implementatie maakt verder nog onderscheid tussen datacellen, builtins (operatoren en constanten) en constructors.
\begin{itemize}
\item
  \texttt{DATA} cellen worden gebruikt om gegevens in op te slaan.
  \emph{IFPL} lijkt onderscheid te maken tussen cellen die getallen opslaan en cellen die lijsten (\texttt{CONS} cellen) opslaan, en eventueel nog aparte structured data cellen.
  Ik zag geen reden om de data cellen bewust te maken van de betekenis van de gegevens die ze dragen, aangezien dit steeds af te leiden valt uit de context.
  Een functie die iets met de data doet, bevat genoeg informatie om de data te interpreteren.
  \texttt{DATA} cellen worden dus zowel gebruikt om primitieve data in op te slaan, zoals integers, of om naar gestructureerde datastructuren te verwijzen.
  In het eerste geval bevat de datacel enkel de waarde van de primitieve data.
  In het tweede geval bevat de cel een \emph{structured data tag} en een pointer naar de datastructuur.
  Wanneer de datastructuur werd aangemaakt door een constructor van een somtype, is het de structured data tag die onthoudt welke constructor dit was.
\item
  Sommige operaties kunnen effici{\"e}nter worden uitgevoerd door een \texttt{BUILTIN} operatie te voorzien in de evaluator.
  Voor arithmetische en logische functies gebruiken we machine-instructies.
  Arithmetische en logische operatoren kunnen ge{\"i}mplementeerd worden door middel van snelle machine-instructies.
  Conditionals en recursie kunnen ook worden uitgedrukt in $\lambda$-calculus, maar met ingebouwde \texttt{IF} en fixed point operatoren vermijden we heel wat overhead.
  Verder kan enkel de \texttt{SELECT} operator velden uit een gestructureerd datatype opvragen.
  \paragraph{}
  TODO: mss dit verplaatsen naar graafreductie section?
  Builtin operatoren verschillen ook van $\lambda$-abstracties in dat ze meer controle hebben over hoe hun argumenten gereduceerd dienen te worden.
  De \texttt{+} operator kan bijvoorbeeld niets zinnigs doen met niet-gereduceerde bomen, en zal dus steeds zijn argumenten volledig trachten te reduceren.
  De \texttt{IF} operator reduceert zijn predicaat eerst volledig, en daarna geeft het het correcte argument ongereduceerd terug.
  \paragraph{}
  Behalve operatoren zijn er ook enkele constanten die als builtins worden ge{\"i}mplementeerd.
  Deze kunnen gezien worden als operatoren met 0 argumenten.
  Een voorbeeld hiervan is de \texttt{FAIL} constante, die gebruikt wordt om een gefaalde poging tot pattern matching te indiceren.
\item
  Hoewel constructors in principe voorgesteld zouden kunnen worden door builtins, heb ik een apart \texttt{CONSTR} celtype gedefinieerd.
  Alle andere ingebouwde procedures hebben een vast aantal parameters per procedure: \texttt{+} heeft er steeds 2, \texttt{IF} heeft er 3, etc.
  Aangezien de gebruiker zijn eigen constructors kan defini{\"e}ren met een arbitrair aantal velden, moet de \texttt{CONSTR} builtin omkunnen met een variabel aantal argumenten.
  Het leek eenvoudiger om hiervoor een apart celtype te introduceren.
\end{itemize}

\section{Implementatie van het reductiealgoritme}
We hebben al vastgesteld dat we een expressie pas wensen te evalueren wanneer haar waarde nodig is, maar we hebben nog geen procedure beschreven hoe we bepalen welke waarde we nodig hebben.
We weten dat ons programma in elk geval uiteindelijk een zekere output moet genereren.
Tot dit eind kunnen we een printing mechanisme voorzien dat het resultaat van ons programma gaat weergeven op de command line.
\paragraph{}
Bij een strikte zuivere taal zouden we het programma eerst volledig reduceren, en daarna het resultaat aan het printing mechanisme geven om zo de output te genereren.
\paragraph{}
Bij onze luie taal loopt dit anders: de print functie gaat de graafreductie aandrijven door de graaf steeds net voldoende te reduceren om het volgende element te kunnen uitprinten.
Indien het resultaat van ons programma een eenvoudig geheel getal is, zien we op dit hoogste niveau niet zo veel verschil: de graaf moet volledig worden gereduceerd tot een enkele \texttt{DATA} cel die een geheel getal bevat.
\paragraph{}
Het wordt interessanter wanneer het resultaat van het programma een lijst is.
Als we in een klassieke imperatieve taal een proces willen beschrijven dat een zeer lange reeks resultaten produceert, dan doen we dat gewoonlijk door middel van een lus die een resultaat produceert en onmiddellijk wegschrijft, waarna het resultaat zelf uit het geheugen verwijderd of overschreven kan worden.
\paragraph{}
In een zuiver functionele taal hebben we deze optie niet, omdat we schrijfacties en berekeningen niet kunnen afwisselen.
Het programma moet dus een lijst met alle resultaten produceren.
Als het printing mechanisme echter vereist dat de volledige lijst wordt ge{\"e}valueerd voordat het begint met resultaten wegschrijven, dan betekent dit dat de volledige lijst eerst in het werkgeheugen van de computer moet worden geplaatst.
Dit is duidelijk een inferieure oplossing in vergelijking met de aanpak van de imperatieve taal: het verbruik van werkgeheugen schaalt lineair met de lengte van de output in het functionele programma!
\paragraph{}
Om dit te vermijden zal de outputprocedure steeds net voldoende reduceren om 1 resultaat te kunnen wegschrijven.
Wanneer een resultaat is weggeschreven, zal de node die het resultaat beschrijft worden losgekoppeld van het deel van de graaf waar we in aan het werken zijn.
Als er geen andere nodes nog verwijzen naar het resultaat, zal in een evaluator met geheugenmanagement dit resultaat kunnen worden verwijderd uit het geheugen.


\subsection{Selecteren van de volgende te reduceren expressie}
\paragraph{}
Een expressie is reduceerbaar wanneer er zich ergens in de graaf een applicatie bevindt met in het linkerlid een $\lambda$-uitdrukking, of een applicatie van een builtin op het correcte aantal argumenten voor die builtin.
\paragraph{}
Zo'n reduceerbare expressie noemen we een \emph{redex}.
\paragraph{}
Een strikt evaluatiealgoritme zal steeds alle redexes reduceren.
Met andere woorden, het brengt de graaf in \emph{normal form}.
\paragraph{}
Zoals we al hebben vastgesteld, wensen we niet noodzakelijk alle redexes te reduceren alvorens controle terug te geven aan het printing mechanisme.
Kortom, we willen normal-order reductie volgen, maar we willen dat de reductie ophoudt wanneer er geen \emph{top-level redex} meer is.
Een graaf die geen top-level redex meer heeft is in \emph{weak head normal form}.
Een expressie is in weak head normal form wanneer ze van de vorm \texttt{F E1 E2 ... En} is met $n \geq 0$, en
\begin{enumerate}
\item \texttt{F} is een variabele of dataobject, OF
\item \texttt{F} is een $\lambda$-abstractie of builtin operator en \texttt{(F E1 E2 ... Em)} is geen redex voor eender welke $m \leq n$.
\end{enumerate}
\paragraph{}
Om even samen te vatten hoe het reductiealgoritme er momenteel uitziet:
\begin{enumerate}
\item
  Het printing mechanisme roept de reduce functie op op de knoop die het aanknopingspunt vormt van het programma.
\item
  De reduce functie reduceert de graaf tot die zich in \emph{weak head normal form (WHNF)} bevindt.
\item
  Indien het resultaat van de reductie een lijst is, gaat het printing mechanisme zichzelf recursief oproepen eerst op het eerste element van de lijst, en dan op de rest van de lijst.
  \paragraph{}
  Indien het resultaat van de reductie een geheel getal (of een ander primitief datatype) is, wordt dit onmiddellijk uitgeprint en is de huidige call van de print procedure klaar.
\end{enumerate}
Omdat de type-informatie van de data uitgewist is tijdens runtime, moet het printing mechanisme in mijn implementatie bij de compilatie op de hoogte worden gebracht wat het type van de output zal zijn (bv ``een lijst van lijsten van integers'').
Wanneer de typechecker ge{\"i}mplementeerd is zal dit automatisch gebeuren, nu moet de informatie nog handmatig worden meegegeven.

\subsection{Hoe de volgende top-level redex te vinden}
De expressie die we wensen te reduceren kan enkel van de vorm \texttt{f E1 E2 ... En} zijn.
We beschouwen alle mogelijkheden voor \texttt{f}:
\begin{enumerate}
\item \texttt{f} is een data-object: De expressie is in WHNF dus we zijn klaar.
\item \texttt{f} is een builtin operator of constructor die $k$ argumenten vereist: De expressie is in WHNF als en slechts als $n < k$. In het andere geval is de regex de expressie \texttt{f E1 ... Ek}. De reductie die erop volgt volgt builtin-specifieke regels.
\item \texttt{f} is een $\lambda$-abstractie: Als er een argument beschikbaar is, dan is \texttt{F E1} de redex.
\end{enumerate}
TODO illustratie, evt figuren uit boek p 201
\paragraph{}


\end{document}
